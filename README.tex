\documentclass{article}
\usepackage[margin=1in]{geometry}
\usepackage{hyperref}
\usepackage[dvipsnames]{xcolor}

\title{Data Logger Project}
\author{Alexander M. Dasneves\\Bridgewater State University\\Bridgewater, MA 02325\and Lyra B. Brown\\Bridgewater State University\\Bridgewater, MA 02325}
\date{\today}


\pagenumbering{gobble}

\begin{document}
  \maketitle

  \section*{Steps Completed}
  \noindent
  We completed steps \textbf{1,2,3,5 \& 7}
  
  \section*{Wiring}
  \noindent
  To wire these devices to the arduino, please connect the wires as such\\
  \begin{tabular}{|c|c|c|}
    \hline
    Device&SDA&SCL\\
    \hline
    GY521&A4&A5\\
    DS1307&A0&A1\\
    \hline
  \end{tabular}
  \vspace{0.1in}
  \\
  \noindent
  \begin{tabular}{|c|c|c|c|c|}
    \hline
    Device&CS&MOSI&MISO&SCK/CLK\\
    \hline
    SC Card Adapter & 10 & 11 & 12 & 13\\
    \hline
  \end{tabular}
  We also added a Debug LED on Pin 3 to help diagnose SD Init Failures.

  \section*{Running Code}
  \noindent
  To run this code, create and source a venv by running the following code:\\
  \begin{verbatim}
python3.10 -m venv venv
source ./venv/bin/activate
  \end{verbatim}
  Then, install the requirements by using the \textit{requirements.txt} file\\
  \begin{verbatim}
python3.10 -m pip install -r requirements.txt
  \end{verbatim}
  Then, run the code using this command\\
  \begin{verbatim}
python3.10 main.py <PORT> <BAUD>
  \end{verbatim}
  The baudrate of our Arduino Sketch uses \textbf{\textit{\underline{9600}}}\\
  \\
  \textit{Please note:} This project uses \textbf{Requests} rather than \textbf{urllib}, as \textbf{Requests} has become the more ubiquitous and commonly implemented library for HTML requests.

  \section*{Results}
  Our code produces a \textit{readings.txt} file on the SD Card connected to the arduino. The file will contain the following information:\\
  \begin{itemize}
    \item Date of reading
    \item Time of reading
    \item Reading Name
    \item Reading value
  \end{itemize}
  This information is also sent down the line through the Serial Connection to the attached computer.\\
  \\
  The Python program provided will interpret the data sent my the Arduino, and perform an HTTP GET request to \href{http://riscy.info/dump.php}{\underline{\textcolor{Emerald}{riscy.info/dump.php}}}, where the readings will be saved.\\\\
  To view all readings collected by the COMP-407 class, visit \href{http://riscy.info/data.txt}{{\underline{\textcolor{Emerald}{riscy.info/data.txt}}}}.
\end{document}

% Hi Maggie!
% Isn't LaTeX Amazing?
